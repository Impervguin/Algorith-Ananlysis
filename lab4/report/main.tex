\ssr{ВВЕДЕНИЕ}

\textbf{Параллелизм} описывает последовательности, которые происходят одновременно \cite{posix-threads}. Таким образом параллельные вычисления таковы, что они выполняются одновременно. Распараллеливание вычислений может привести к росту временной эффективности программы при использовании на многопроцессорных и однопроцессорных, если программа часто блокируется ожиданиями ввода/вывода, системах \cite{tanenbaum}.

В современных системах различают параллельность реализованную на потоках и процессах. При этом стоит учитывать, что часто программы выполняются не строго параллельно, а конкурентно, то есть выполняются последовательно, постоянно переключаясь между друг другом \cite{tanenbaum}.

Целью данной работы является разработка ПО, выполняющего скачивание страниц и парсинг рецептов с сайта menunedeli.ru.

Задачами работы являются:

\begin{itemize}
	\item рассмотрение структуры сайта;
	\item разработка ПО, выполняющего скачивание и парсинг в одном потоке;
	\item разработка многопоточного ПО, выполняющего скачивание и парсинг;
	\item исследование временных характеристик разработанных программ.
\end{itemize}
\vspace{20mm}
{\let\clearpage\relax \chapter{Входные и выходные данные}}

Входными данными для программы являются ссылки на страницы сайта menunedeli.ru. с рецептами. Каждая ссылка содержит один рецепт в веб-ресурсе. Выходными данными являются файлы -- загруженные рецепты по ссылкам из входных данных, при этом в файлах содержится часть сайта с рецептом.

\vspace{20mm}
{\let\clearpage\relax \chapter{Преобразование входных данных в выходные}}

Для получения ссылок с рецептами с сайта в виде файла разработан скрипт parse.py на языке python3:

\begin{lstlisting}[label=parse,caption={parse.py -- скрипт для получения ссылок с рецептами,}]
	import requests as re
	import bs4
	import argparse
	import sys
	
	parser = argparse.ArgumentParser(
	prog="parse",
	description="Скачивает ссылки на статьи с сайта menunedeli.ru"
	)
	parser.add_argument("-c", "--count", type=int, default=1000, help="Количество ссылок для скачивания")
	parser.add_argument("-s", "--save", type=str, default="links.txt", help="Файл, куда сохранять ссылки")
	
	args = parser.parse_args()
	
	saveTo = args.save
	UpToLinks = args.count
	catalogFormat = "https://menunedeli.ru/novye-stati/page/{}"
	with open(saveTo, "w") as f:
	links = 0
	page = 1
	while True:
	catalogPage = re.get(catalogFormat.format(page))
	
	bs = bs4.BeautifulSoup(catalogPage.content, "lxml")
	
	for a in bs.find_all('article'):
	for link in a.find_all('meta', attrs={'itemprop':'url'}):
	print(link['content'], file=f)
	links += 1
	if links >= UpToLinks:
	exit(0)
	page += 1
\end{lstlisting}

Полученный в результате работы программы \ref{parse} файл выступает входными данными для разработанного ПО.

Веб-страница по каждой из ссылок скачивается и обрабатывается одной из программ либо в однопоточном режиме, либо многопоточном, при этом из всей полученной страницы остаётся только часть с рецептом.

В результате работы программ для каждой из ссылок создаётся html файл, в котором хранится рецепт.

Программы, выполняющие скачивание и обработку страниц был использован язык программирования C \cite{C}, при этом для реализации многопоточности были использованы posix threads из библиотеки pthreads \cite{pthreads}.

\vspace{20mm}
{\let\clearpage\relax \chapter{Примеры работы программы}}

Один из примеров обработки веб-страниц \cite{receipt} представлен на рисунках \ref{before}-\ref{after}.

\begin{figure}[h]
	\centering
	\includegraphics[width=0.9\textwidth]{before}
	\caption{Веб-страница по ссылке \cite{receipt}}
	\label{before}
\end{figure}

\begin{figure}[h]
	\centering
	\includegraphics[width=0.9\textwidth]{myreceipt}
	\caption{Результат обработки программой}
	\label{after}
\end{figure}

\vspace{20mm}
{\let\clearpage\relax \chapter{Тестирование}}

Тестирование программы проводилось загрузкой в качестве входных данных 1000 ссылок на различные рецепты веб-ресурса menunedeli.ru. При этом отслеживалось количество удачных скачиваний и обработок страницы, а также выборочная ручная проверка 10 случайных выходных файлов. 

Тестирование проводилось как для однопоточной реализации, так и для многопоточной, при этом выходные файлы обоих реализаций при тестировании оказались идентичными.

Тестирование было успешно пройдено.

\vspace{20mm}
{\let\clearpage\relax \chapter{Описание исследования}}

В ходе исследования были собраны данные о временных характеристиках однопоточной и многопоточной реализаций, при этом для многопоточной реализации были собраны данные при количестве потоков равному 1, 2, 4, 8, 16, 32, 64, 128.

Для каждого из вариантов программы выполнение проводилось 10 раз и бралось среднее арифметическое времён, на вход подавался файл с 1000 ссылок.

Технические характеристики устройства,на котором проводились замеры:

\begin{itemize}
	\item процессор: AMD Ryzen 7 5800H (16) @ 4.46 ГГц;
	\item оперативная память: 16 ГБ;
	\item операционная система: Arch Linux x86\_64.
\end{itemize}

При проведении замеров ноутбук был включён в сеть и были запущены только системные приложения.

Временные характеристики реализаций представлены в таблице \ref{tbl:time} и на графике \ref{plot}.

\begin{figure}[h]
	\centering
	\includegraphics[width=0.9\textwidth]{plot}
	\caption{График временных характеристик разработанных программ}
	\label{plot}
\end{figure}

\begin{longtable}{|r|r|}
	\caption{Временные характеристики разработанных программ (начало)}\label{tbl:time}
	\\
	\hline
	Количество потоков & Время работы, с \\
	\hline
	\endfirsthead
	\caption{Временные характеристики разработанных программ (окончание)}
	\\
	\hline
	Количество потоков & Время работы, с \\
	\hline
	\endhead
	\hline
	\endfoot
	\endlastfoot
	\hline
	\multicolumn{2}{|c|}{Однопоточная реализация} \\
	\hline
	1 & 164.98 \\
	\hline
	\multicolumn{2}{|c|}{Многопоточная реализация} \\
	\hline
	1 & 169.45 \\
	\hline
	2 & 90.11 \\
	\hline
	4 & 50.85 \\
	\hline
	8 & 31.79 \\
	\hline
	16 & 25.78 \\
	\hline
	32 & 27.32 \\
	\hline
	64 & 28.42 \\
	\hline
	128 & 114.53 \\
	\hline
\end{longtable}

В результате исследования сделан вывод, что использование параллельных вычислений на нативных потоках может ускорить программу, однако использование потоков, больше числа потоков процессора (в случае устройства, на котором проводились замеры >16 потоков), не даст увеличения временной производительности.

\clearpage

\ssr{ЗАКЛЮЧЕНИЕ}

Целью -- разработка ПО, выполняющего скачивание страниц и парсинг рецептов с сайта menunedeli.ru -- была выполнена.

В ходе работы были решены следующие задачи:
\begin{itemize}
	\item рассмотрена структура сайта;
	\item разработано ПО, выполняющее скачивание и парсинг в одном потоке;
	\item разработано многопоточное ПО, выполняющее скачивание и парсинг;
	\item исследованы временные характеристики разработанных программ.
\end{itemize}

В ходе исследования было выявлено, что использование параллельных вычислений на нативных потоках ускоряет программу, однако использование большего числа потоков, чем может выполнять процессор не даёт прироста производительности, а замедляет программу.






