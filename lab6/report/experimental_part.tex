\chapter{Исследовательская часть}

\section{Параметризация}

Параметризация проводится по 3-м параметрам муравьиного алгоритма: $\alpha$ -- коэффициент влияние феромонов на выбор муравья, $p$ -- коэффициент испарения феромонов, дни -- количество дней в муравьином алгоритме.

\subsection{Класс данных}

В качестве класса данных для параметризации используем графы, построенные на городах Африки, при этом всего будет 3 графа. В качестве весов графа использовались расстояния между этими городами по прямой в км.

\subsubsection{Граф 1}
\begin{enumerate}
	\item Каир (Египет);
	\item Лагос (Нигерия);
	\item Найроби (Кения);
	\item Кейптаун (Южноафриканская Республика);
	\item Аккра (Гана);
	\item Дакар (Сенегал);
	\item Абиджан (Кот-д'Ивуар);
	\item Дар-эс-Салам (Танзания);
	\item Аддис-Абеба (Эфиопия);
	\item Хартум (Судан).
\end{enumerate}

\subsubsection{Граф 2}
\begin{enumerate}
	\item Бондио (Габон);
	\item Нджамена (Чад);
	\item Конакри (Гвинея);
	\item Луанда (Ангола);
	\item Мансу (Сьерра-Леоне);
	\item Джамбул (Казахстан);
	\item Махаджанга (Мадагаскар);
	\item Габороне (Ботсвана);
	\item Виндхук (Намибия);
	\item Ямусукро (Кот-д'Ивуар).
\end{enumerate}

\subsubsection{Граф 3}
\begin{enumerate}
	\item Тунис (Тунис);
	\item Кигали (Руанда);
	\item Момбаса (Кения);
	\item Лусака (Замбия);
	\item Джуба (Южный Судан);
	\item Мапуту (Мозамбик);
	\item Бамако (Мали);
	\item Либревиль (Габон);
	\item Кигали (Руанда);
	\item Антананариву (Мадагаскар).
\end{enumerate}

\subsection{Результаты параметризации}

В результате параметризации оказалось, что самыми лучшими параметрами по заданному классу данных оказались при $\alpha = 0.3$, $p = 0.3$ и количеством дней равным 200. При этом данные параметры дают лучший результат как по средним параметрам, так и по минимальным значениям. Результаты параметризации для данных параметра представлены в таблице~(\ref{tbl:d}), а вся таблица параметризации представлена в приложении А.

\begin{longtable}{|r|r|r|r|r|r|r|r|r|r|r|r|}
	\caption{Результаты параметризации муравьиного алгоритма}\label{tbl:d}
	\\
	\hline
	\multicolumn{3}{|c|}{Параметры} & \multicolumn{3}{|c|}{Граф 1} & \multicolumn{3}{|c|}{Граф 2} & \multicolumn{3}{|c|}{Граф 3} \\
	\hline
	\multicolumn{1}{|c|}{$\alpha$} & \multicolumn{1}{|c|}{p} & \multicolumn{1}{|c|}{Дни} & \multicolumn{1}{|c|}{min} & \multicolumn{1}{|c|}{max} & \multicolumn{1}{|c|}{avg} & \multicolumn{1}{|c|}{min} & \multicolumn{1}{|c|}{max} & \multicolumn{1}{|c|}{avg} & \multicolumn{1}{|c|}{min} & \multicolumn{1}{|c|}{max} & \multicolumn{1}{|c|}{avg} \\
	\endfirsthead
	\hline
	0.3 & 0.3 & 200 & 250.90 & 250.90 & 250.90 & 225.50 & 233.50 & 226.30 & 258.60 & 261.60 & 258.90\\
	\hline
\end{longtable}

\section{Вывод}

В результате исследовательской части была проведена параметризация по заданному классу данных и выявлены лучшие параметры.

\clearpage
