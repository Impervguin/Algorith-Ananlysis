\chapter{Технологическая часть}
\section{Средства разработки}

В качестве языка программирования был выбран go~\cite{go}, так как данный язык обладает достаточными средствами для реализации алгоритмов, а также позволяет реализовать конкурентно отдельные части алгоритма с помощью горутин~\cite{go-mem}.

\section{Реализация алгоритмов}

В листинге~(\ref{full-search}) представлен алгоритма полного перебора, при этом часть генерации перестановок и расчёта длины пути выполняются конкурентно в горутинах, передавая данные через канал~\cite{go-mem}.

\begin{lstlisting}[label=full-search,caption={Алгоритм полного перебора},language=go]
func HeapAlgo(arr []int) <-chan []int {
	ch := make(chan []int)
	go func() {
		c := make([]int, len(arr))
		for i := range c {
			c[i] = 0
		}
		arrCopy := make([]int, len(arr))
		
		copy(arrCopy, arr)
		ch <- arrCopy
		
		i := 1
		for i < len(arr) {
			if c[i] < i {
				if i%2 == 0 {
					arr[0], arr[i] = arr[i], arr[0]
				} else {
					arr[c[i]], arr[i] = arr[i], arr[c[i]]
				}
				copy(arrCopy, arr)
				ch <- arrCopy
				c[i] += 1
				i = 1
			} else {
				c[i] = 0
				i++
			}
		}
		close(ch)
	}()
	return ch
}

type FullSearch struct{}

func NewFullSearch() *FullSearch {
	return &FullSearch{}
}

func (f *FullSearch) Run(gr *graph.WeightedUndirectedGraph) (*graph.WeightedCycle, error) {
	bestCycle := gr.GetRandomHamiltonian()
	ch := HeapAlgo(gr.GetNodes())
	for arr := range ch {
		path := graph.NewWeightedCycle(gr)
		for _, node := range arr {
			err := path.AddNode(node)
			if err != nil {
				return nil, err
			}
		}
		if path.CalculateWeight() < bestCycle.CalculateWeight() {
			bestCycle = path
		}
	}
	return bestCycle, nil
}
\end{lstlisting}

В листинге~(\ref{ant-algo}) представлена реализация муравьиного алгоритма, а на~(\ref{ant-strategy}) реализация построения пути отдельным муравьём.

\begin{lstlisting}[label=ant-algo,caption={Муравьиный алгоритм},language=go]
type ElitistAntAlgorithm struct {
	distanceCoeff    float64
	pheromoneCoeff   float64
	evaporationCoeff float64
	initPheromone    float64
	pheromonePerAnt  float64
	
	antsCount      int
	eliteAntsCount int
	daysCount      int
}

func NewElitistAntAlgorithm(distanceCoeff, pheromoneCoeff, evaporationCoeff, initPheromone, pheromonePerAnt float64, antsCount, eliteAntsCount, daysCount int) *ElitistAntAlgorithm {
	return &ElitistAntAlgorithm{
		distanceCoeff:    distanceCoeff,
		pheromoneCoeff:   pheromoneCoeff,
		evaporationCoeff: evaporationCoeff,
		initPheromone:    initPheromone,
		pheromonePerAnt:  pheromonePerAnt,
		
		antsCount:      antsCount,
		eliteAntsCount: eliteAntsCount,
		daysCount:      daysCount,
	}
}

func (a *ElitistAntAlgorithm) Run(gr *graph.WeightedUndirectedGraph) (*graph.WeightedCycle, error) {
	bestCycle := gr.GetRandomHamiltonian()
	
	phgr := NewGraphWithPheromon(gr, a.initPheromone)
	for day := 0; day < a.daysCount; day++ {
		ants := make([]*Ant, a.antsCount)
		initNode := gr.GetNodes()[0]
		for i := 0; i < a.antsCount; i++ {
			ants[i] = NewAnt(phgr, a.pheromonePerAnt, a.distanceCoeff, a.pheromoneCoeff, initNode)
		}
		
		for _, ant := range ants {
			if err := ant.Go(); err != nil {
				return nil, err
			}
			
			if ant.GetPath().CalculateWeight() < bestCycle.CalculateWeight() {
				bestCycle = ant.GetPath()
			}
		}
		
		phgr.EvaporatePheromone(a.evaporationCoeff)
		
		for _, ant := range ants {
			phgr.ApplyPheromon(ant.GetPath(), ant.pheromone)
		}
		
		for i := 0; i < a.eliteAntsCount; i++ {
			phgr.ApplyPheromon(bestCycle, a.pheromonePerAnt)
		}
	}
	return bestCycle, nil
}
\end{lstlisting}

\begin{lstlisting}[label=ant-strategy,caption={Алгоритм построения пути муравьём},language=go]
func (a *Ant) chooseNextNode() int {
	lastNode, _ := a.path.LastNode()
	
	sumDesire := 0.
	for _, node := range a.unvisited {
		sumDesire += a.desireFunc(lastNode, node)
	}
	
	probabilities := make(map[int]float64, len(a.unvisited))
	for i := range a.unvisited {
		probabilities[i] = a.desireFunc(lastNode, a.unvisited[i]) / sumDesire
	}
	randVal := rand.Float64()
	sumProb := 0.
	for i, prob := range probabilities {
		sumProb += prob
		if sumProb >= randVal {
			node := a.unvisited[i]
			a.unvisited = append(a.unvisited[:i], a.unvisited[i+1:]...)
			return node
		}
	}
	return -1
}

func (a *Ant) Go() error {
	for len(a.unvisited) > 0 {
		nextNode := a.chooseNextNode()
		if nextNode == -1 {
			return fmt.Errorf("no valid next node found")
		}
		a.path.AddNode(nextNode)
	}
	return nil
}
\end{lstlisting}

\section{Оценка сложности алгоритмов}

Для алгоритма полного перебора оценка сложности составляет $O(n!*n) = O(n!)$, так как сложность алгоритма генерации перестановок $O(n!)$~\cite{perms-methods}, а сложность поиска цены пути составляет $O(n)$.

У муравьиного алгоритма оценка сложности рассчитывается следующим образом

\begin{equation}
	\label{eq:ant-o}
	O(d * (m * (n^2) + (n^2) + m*n + n)) = O(dmn^2),
\end{equation}
где
\begin{itemize}
	\item d -- количество дней;
	\item m -- число муравьёв;
	\item n -- число узлов.
\end{itemize}

\section{Тестирование}

Тестирование алгоритма полного перебора  возможно провести с проверкой точного значения, однако для муравьиного алгоритма можно проверить лишь то, что результирующий цикл не содержит все узлы, то есть является гамильтоновым циклом.

\subsubsection{Тест 1}

\textbf{Входные данные:}
\[
\begin{bmatrix}
	0 & 10 & 15 & 20 & 25 \\
	10 & 0 & 30 & 35 & 40 \\
	15 & 30 & 0 & 45 & 50 \\
	20 & 35 & 45 & 0 & 55 \\
	25 & 40 & 50 & 55 & 0 \\
\end{bmatrix}
\]

\textbf{Ожидаемое значение:} [0 1 2 3 4], 100.0


\textbf{Выходные данные полного перебора:} [0 1 2 3 4], 100.0


\textbf{Выходные данные муравьиного алгоритма:} [0 4 3 2 1] 100.0


\subsubsection{Тест 2}

\textbf{Входные данные:}
\[
\begin{bmatrix}
	0.00 & 12.34 & 23.45 & 34.56 & 45.67 & 56.78 & 67.89 & 78.90 & 89.01 & 90.12 \\
	12.34 & 0.00 & 13.14 & 25.35 & 36.46 & 47.57 & 58.68 & 69.79 & 80.90 & 91.01 \\
	23.45 & 13.14 & 0.00 & 14.25 & 26.36 & 38.47 & 50.58 & 62.69 & 74.80 & 86.91 \\
	34.56 & 25.35 & 14.25 & 0.00 & 16.37 & 28.48 & 40.59 & 52.70 & 64.81 & 76.92 \\
	45.67 & 36.46 & 26.36 & 16.37 & 0.00 & 18.49 & 30.60 & 42.71 & 54.82 & 66.93 \\
	56.78 & 47.57 & 38.47 & 28.48 & 18.49 & 0.00 & 12.61 & 24.72 & 36.83 & 48.94 \\
	67.89 & 58.68 & 50.58 & 40.59 & 30.60 & 12.61 & 0.00 & 12.83 & 24.94 & 37.05 \\
	78.90 & 69.79 & 62.69 & 52.70 & 42.71 & 24.72 & 12.83 & 0.00 & 13.05 & 25.16 \\
	89.01 & 80.90 & 74.80 & 64.81 & 54.82 & 36.83 & 24.94 & 13.05 & 0.00 & 13.27 \\
	90.12 & 91.01 & 86.91 & 76.92 & 66.93 & 48.94 & 37.05 & 25.16 & 13.27 & 0.00 \\
\end{bmatrix}
\]


\textbf{Ожидаемое значение:} [0 1 2 3 4 5 6 7 8 9] 216.47


\textbf{Выходные данные полного перебора:} [0 1 2 3 4 5 6 7 8 9] 216.47


\textbf{Выходные данные муравьиного алгоритма:} [0 1 2 3 4 6 7 8 9 5] 231.57

\section*{Вывод}

В ходе технологической части работы были разработаны муравьиный алгоритм и алгоритм полного перебора для задачи коммивояжёра, а также проведена оценка их временной сложности. Все тесты успешно пройдены.

\clearpage
