\ssr{ВВЕДЕНИЕ}

\textbf{Задача коммивояжёра} – одна из самых популярных и старинных задач комбинаторной оптимизации. Впервые задача была исследована в 18-м веке ирландским математиком Уильямом Роуан Гамильтоном и британским математиком Томасом Пенингтон Киркманом в их книге~\cite{tsp-donald} "Graph Theory".

Целью данной работы является рассмотрение методов решения задачи коммивояжёра.

Для достижения этой цели требуется решить следующие задачи:

\begin{itemize}
	\item постановка задачи коммивояжёра;
	\item рассмотрение методов полного перебора и муравьиной колонии;
	\item разработка рассмотренных алгоритмов решения задачи;
	\item реализация разработанных алгоритмов;
	\item проведение параметризации для алгоритма на основе метода муравьиной колонии. 
\end{itemize}

\newpage

\clearpage
