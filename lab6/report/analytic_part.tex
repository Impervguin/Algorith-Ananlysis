\chapter{Аналитическая часть}
\section{Постановка задачи}

Пусть дан неориентированный полносвязный граф $G_n = (V_n, E_n)$ с $n = |V_n|$ узлов и $m = |E_n| = C^2_n$ рёбер. Ребро с концами i и j обозначается как ij или как (i, j). 

Пусть также даётся функция $f:E_n -> \mathbf{R}$ разметки графа. Каждая значение функции называется длиной ребра (i, j) и пусть длиной цикла называется сумма длин всех рёбер входящих в него.

Гамильтоновым циклом в неориентированном графе называется цикл графа, проходящий через каждую его вершину~\cite{graphs}.

Тогда задача поиска гамильтонова цикла с минимальной длинной называется симметричной задачей коммивояжёра~\cite{tsp}.

\section{Методы решения}

В данной работе будут рассмотрены 2 метода решения задачи коммивояжёра: метод полного перебора и метод муравьиной колонии

\subsection{Метод полного перебора}

Под методом полного перебора подразумевается решение задачи, при котором формируются все возможные гамильтоновы циклы, просчитывается их длины и выбирается тот, у которого эта длина минимальна. Метод полного перебора в результате даёт оптимальное решение, тем не менее с увеличением числа узлов в графе сложность метода существенно возрастает, так как для графа из $n$ узлов существует $\frac{(n-1)!}{2}$~\cite{tsp} гамильтоновых циклов. Таким образом фактическая трудоёмкость алгоритма полного перебора возрастает не медленнее чем по факториальной зависимости.

\subsection{Метод муравьиной колонии}

Метод муравьиной колонии -- один из стохастических методов, применяющихся для решения задач комбинаторной оптимизации, в частности для решения задачи коммивояжёра.

Метод основан на способе обмена информацией у муравьев, при котором они передают информацию о привлекательности путей с помощью меток(феромонов).

Не считая этап инициализации метод состоит из основного цикла, который делится на 2 части~\cite{heurestics}.

\subsubsection{Конструирование путей муравьёв}

На данном этапе каждый из $m$ муравьёв инициализируется с циклом с единственным начальным узлом $c_p$ и на каждом шаге стохастически выбирает узел из тех, которые он не посещал, и добавляет в свой цикл.

Выбор узла на очередном этапе цикла производится с вероятностью, вычисляемой по формуле~(\ref{eq:ant-chance})~\cite{heurestics}.

\begin{equation}
	\label{eq:ant-chance}
	p(e^i_j) = \frac{\tau^\alpha_{ij}*f(e^i_j)^\beta}{\sum{\tau^\alpha_{ij}*f(e^i_j)^\beta}},
\end{equation}
где 
\begin{itemize}
	\item $i$ -- номер последнего добавленного узла в пути;
	\item $j$ -- номер рассматриваемого узла из не посещённых муравьём узлов;
	\item $e^i_j$ -- ребро между i-м и j-м узлами;
	\item $\tau_{ij}$ -- количество феромона на ребре между i-м и j-м узлами;
	\item $f(e^i_j)$ -- функция выражающая "привлекательность" ребра между i-м и j-м узлами для муравья. В случае задачи коммивояжёра эта величина обратна пропорциональная длине ребра и выражается формулой $\frac{1}{L(e^i_j)}$;
	\item $\alpha$ и $\beta$ $\in (0, 1)$ -- параметры определяющие влияние феромонов и длины пути на вероятность выбора ребра. При $\alpha = 0$ выбор муравья полностью жадный и определяется длиной ребра, а при $\beta = 0$ выбор муравья полностью стайный и определяется только количеством феромона на ребре.
\end{itemize}

В результате этапа каждый муравей формирует гамильтонов цикл по графу. Каждый из сформированных  циклов сравнивается с лучшим на текущий момент.

\subsubsection{Обновление феромонов}

На данном этапе формируются положительные и отрицательные обратные связи. Каждый муравей проходя по своему пути распыляет на нём феромоны, при этом чем длиннее путь, тем меньше распылённое количество феромонов. Распылённое число феромонов для k-го муравья рассчитывается по формуле~(\ref{eq:pherplus})~\cite{ieee}.

\begin{equation}
	\label{eq:pherplus}
	\triangle\tau_{ij}^k = \begin{cases}
		\frac{Q}{L_k}, &\text{если ребро (i, j) есть в цикле муравья}, \\
		0, &\text{иначе},
	\end{cases}
\end{equation}
где $L_k$ -- длина цикла k-го муравья, а $Q$ -- константа.

Для того, чтобы количество феромона не было решающим фактором после нескольких циклов в модели есть испарение феромона, которое в конце цикла уменьшает концентрацию феромона на каждом ребре. Таким образом длина рёбер продолжает влиять на решение и одновременно наиболее длинные пути становятся менее привлекательными для муравья. Испарение феромона рассчитывается по формуле~(\ref{eq:pherminus})~\cite{ieee}.

\begin{equation}
	\label{eq:pherminus}
	\tau_{ij} = (1-p)\tau_{ij},
\end{equation}
где $p$ -- коэффициент испарения.

Таким образом феромон изменяется по формуле~(\ref{eq:phermon})

\begin{equation}
	\label{eq:phermon}
	\tau_{ij} = (1-p)\tau_{ij} + \sum_{k=0}^{m}\triangle\tau_{ij}^k.
\end{equation}

\subsubsection{Элитарные муравьи}

Элитарные муравьи -- модификация муравьиного алгоритма. Элитарные муравьи на каждой итерации идут по лучшему найденному пути и распыляют по нему феромоны, таким образом увеличиваю привлекательность его частей для других муравьёв~\cite{ieee}. С данной модификацией расчёт феромона на каждой итерации происходит по формуле~(\ref{eq:elitephermon}).

\begin{equation}
	\label{eq:elitephermon}
	\tau_{ij} = (1-p)\tau_{ij} + \sum_{k=0}^{m}\triangle\tau_{ij}^k +me*\frac{Q}{L_b},
\end{equation}
где $me$ --  количество элитных муравьёв, $L_b$ -- длина лучшего цикла.

\section*{Вывод}

В результаты аналитического раздела была рассмотрена формальная постановка задачи коммивояжёра, а также рассмотрены методы её решения.


\clearpage
