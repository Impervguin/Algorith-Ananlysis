\ssr{ВВЕДЕНИЕ}

\textbf{Графовая модель} -- ориентированный конечный граф, вершины которого являются некоторыми командами или участками исполняемого кода, а дуги выражают некоторое отношение между этими участками. Таким образом с помощью графовой модели могут быть выражены отношения последовательности, зависимости по данным и другие. Таким модели могут использоваться для анализа разработанных программ, в частности для поиска участков программы, которые могут быть выполнены параллельно.

Целью данной работы является разработка графовых моделей для заданного участка кода программы.

Для достижения цели необходимо выполнить следующие задачи:
\begin{itemize}
	\item рассмотрение участка кода;
	\item разработка 4-х графовых моделей: графа управления, информационного графа, операционной истории и информационной истории.
\end{itemize}
\vspace{20mm}
{\let\clearpage\relax \chapter{Используемая программа}}

В листинге~\ref{code} представлена анализируемая программа, написанная на языке Go.

\begin{lstlisting}[label=code,caption={Данные, передающиеся между частями конвейера},language=go]
	
	package lab7
	
	import (
	"golang.org/x/net/html"
	)
	
	type TokenNode struct {
		StartTag *html.Token
		Children []*TokenNode
		EndTag   *html.Token
	}
	
	type TokenForest struct {
		Trees []*TokenNode
	}
	
	func (forest *TokenForest) Find(tag string, attrs map[string]string) *TokenNode { // 1
		var result *TokenNode
		for _, tree := range forest.Trees { // 2
			result = tree.Find(tag, attrs) // 3
			if result != nil {             // 4
				return result // 5
			}
		}
		return nil // 6
	}
	
	func (root *TokenNode) Find(tag string, attrs map[string]string) *TokenNode { // 7
		if root.StartTag.Data == tag { // 8
			if attrs == nil { // 9
				return root // 10
			}
			for _, attr := range root.StartTag.Attr { // 11
				if attrs[attr.Key] == attr.Val { // 12
					return root // 13
				}
			}
		}
		for _, child := range root.Children { // 14
			result := child.Find(tag, attrs) // 15
			if result != nil {               // 16
				return result // 17
			}
		}
		return nil // 18
	}
\end{lstlisting}

Программа~\ref{code} описывает две структуры:
\begin{itemize}
	\item $TokenNode$ -- структура, описывающая узел синтаксического дерева html. Состоит из начально и конечного синтаксических единиц, а также из указателей на узлы, которые находятся внутри текущего;
	\item $TokenForest$ -- лес html деревьев. Состоит из массива $TokenNode$ -- корней деревьев.
\end{itemize}

Также в программе представлены два метода:

\begin{itemize}
	\item $Find$ структуры $TokenNode$ -- поиск первого тега в дереве, содержащего хотя бы один из атрибутов заданных ключом и значением;
	\item $Find$ структуры $TokenNode$ -- такой же поиск, но в лесу деревьев.
\end{itemize}

Комментарии в методах после определённых строк представляют номера команд, которые будут использованы при построении графовых моделей.

\chapter{Графовые модели}

\section{Граф управления}

Граф управления представляет собой описание передачи управления в программе. Таким образом дуги в таком описании показывают, какие команды могут исполняться непосредственно друг за другом. граф управления для программы~\ref{code} представлен на рисунке~\ref{fig:GU}.

\begin{figure}[H]
	\centering
	\includesvg[width=0.9\textwidth]{graphs/GU}
	\caption{Граф управления}
	\label{fig:GU}
\end{figure}

\section{Информационный граф}

Информационный граф описывает передачу данных между командами. На нём описываются какие данные передаются от команды к командам. Информационный граф для программы~\ref{code} представлен на рисунке~\ref{fig:IG}.

\begin{figure}[H]
	\centering
	
	\includesvg[width=0.9\textwidth]{graphs/IG}
	\caption{Информационный граф}
	\label{fig:IG}
\end{figure}

\section{Операционная история}

Операционная история, как и граф управления, описывает отношение управления, однако в отличие от него операционная история содержит информацию о некотором запуску программы, где каждая вершина имеет не более одного входа и одного выхода. Операционная история для программы~\ref{code} представлен на рисунке~\ref{fig:OH}.

\begin{figure}[H]
	\centering
	
	\includesvg[width=0.9\textwidth]{graphs/OH}
	\caption{операционная история}
	\label{fig:OH}
\end{figure}

\section{Информационная история}

Информационная история, как и операционная история представляют собой некоторый запуск программы, однако он содержит информацию о последовательности команд, но содержит информацию о том, какие команды требуют информацию от других команд. Информационная история для программы~\ref{code} представлен на рисунке~\ref{fig:IH}.

\begin{figure}[H]
	\centering
	
	\includesvg[width=0.9\textwidth]{graphs/IH}
	\caption{Информационная история}
	\label{fig:IH}
\end{figure}

\clearpage

\ssr{ЗАКЛЮЧЕНИЕ}

Цель -- разработка графовых моделей для заданного участка кода программы -- была выполнена.

В ходе работы была описана используемая программа, а также разработаны 4 графовые модели, описывающие её: граф управления, информационный графа, операционная история и информационная история.


Разработанные графовые модели действительно упрощают анализ программ. Граф управления и информационный граф помогают обнаружить скрытые связи между участками кода, а операционная история и информационная история позволяют обнаружить независимые части, которые потенциально могут выполняться параллельно.






