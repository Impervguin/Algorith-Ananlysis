\ssr{ВВЕДЕНИЕ}

\textbf{Расстояние Левенштейна} \cite{levenshtein} – минимальное количество операций вставки одного символа, удаления одного символа и замены одного символа на другой, необходимых для превращения одной строки в другую.

Расстояние Левенштейна применяется в теории информации и компьютерной лингвистике для:

\begin{itemize}
	\item исправления ошибок в слове
	\item сравнения текстовых файлов утилитой diff
	\item в биоинформатике для сравнения генов, хромосом и белков
\end{itemize}

Целью данной лабораторной работы является изучение метода динамического программирования на материале алгоритмов
Левенштейна и Дамерау-Левенштейна. 

Задачами данной лабораторной являются:
\begin{enumerate}
	\item Рассмотрение алгоритмов Левенштейна и Дамерау-Левенштейна нахождения расстояния между строками;
	\item Разработка рекурсивного алгоритма расстояния Левенштейна, а также применение методов динамического программирования для разработки алгоритмов расстояний 
	
	Левенштейна и Дамерау-Левенштейна с использованием кеширования; 
	\item Реализация разработанных алгоритмов; 
	\item Исследование различий рекурсивной и
	итерационной реализаций алгоритма определения расстояния Левенштейна во временной и ёмкостной характеристиках с помощью замеров процессорного времени и пиковой памяти на строках варьирующейся длины.
\end{enumerate}
\newpage

\clearpage
