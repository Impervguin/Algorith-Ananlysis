\begin{center}
	\LARGE\bfseries{РЕФЕРАТ}
\end{center}

Расчетно-пояснительная записка \pageref{LastPage} с., \totalfigures{} рис., \totaltables{} таблиц, 27 источников, 1 приложение.

ШАХМАТЫ, КУБОК МИРА ПО ШАХМАТАМ, СТАВКИ, БАЗЫ ДАННЫХ, ЛОГИРОВАНИЕ, КЭШИРОВАНИЕ, POSTGRESQL, INFLUXDB, REDIS

Цель работы~--- разработка базы данных сыгранных на кубке мира шахматных партий.

В рамках курсовой работы была разработана база данных сыгранных на кубке мира шахматных партий и приложение к ней. Был проведен анализ предметной области, связанной с проведением кубка мира по шахматам и ставками на спорт. Были рассмотрены и сравнены существующие решения для хранения шахматных партий. Были сформулированы требования к проектируемым программному обеспечению и базе данных. Были рассмотрены системы управления базами данных на основе формализованной задачи. Были описаны сущности проектируемой базы данных и пользователи разрабатываемого приложения.

Были формализованы бизнес-правила приложения и спроектирована база данных. Были описаны ролевая модель и ограничения базы данных. Были разработаны схемы алгоритмов триггеров, необходимых для корректной работы системы. Была описана структура разрабатываемого приложения.

Были проанализированы и выбраны средства реализации приложения и базы данных. Были описаны триггеры, пользователи и ограничения целостности базы данных. Был разработан графический пользовательский интерфейс приложения.

Было проведено исследование, целью которого являлось определение зависимости среднего времени получения результата запроса на стороне фронтенда от параметра TTL кэша. По результатам измерений можно сделать вывод, что время получения результата запроса уменьшается по мере увеличения параметра TTL кэша, при этом при TTL большем 1000 мс наблюдается насыщение.
