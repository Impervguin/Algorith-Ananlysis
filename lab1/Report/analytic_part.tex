\chapter{Аналитическая часть}
\section{Расстояние Левенштейна}
\textbf{Расстояние Левенштайна} - метрика, измеряющая редакторское расстояние между двумя последовательностями символов. Расстояние Левенштайна - минимальное количество операций вставки, замены и удаления символов, необходимых для преобразования одной последовательности в другую. Стоимости каждого вида операций могут быть различны, однако в данной работе примем их всех равными единице, то есть будем считать количество операций, требуемых для преобразования.

\subsection{Рекурсивная формула расстояния Левенштейна}

Пусть существует две последовательности символов $S_1$ и $S_2$ длины N и M соответсвенно. Ввёдем функцию $D(i, j)$, равную расстоянию левенштейна между подстроками $S_1[1...i]$ и $S_2[1...j]$.

Тогда расстояние Левенштайна для двух строк может быть выражено следующей реккурентной формулой:

\begin{equation}
	\label{eq:D}
	D(i, j) = 
	\begin{cases}
		0 &\text{i = 0 и j = 0}\\
		i &\text{i > 0 и j = 0}\\
		j &\text{i = 0 и j > 0}\\
		min \begin{cases}
			D(i, j - 1) + 1 &\text{Вставка}\\
			D(i - 1, j) + 1 &\text{Удаление}\\
			D(i - 1, j - 1) + f_e(S_1[i], S_2[j]) &\text{Замена}\\
		\end{cases} &\text{i > 0 и j > 0}\\
	\end{cases}
\end{equation}
При этом
\begin{equation}
	\label{eq:f_e}
	f_e(a1, a2) = 
	\begin{cases}
		0 &\text{a1 == a2}\\
		1 &\text{a1 != a2}\\
	\end{cases}
\end{equation}
Исходя из функции можно понять, что она просчитывает минимальное расстояние Левенштейна к текущему моменту(индексу в первой строке), пытаясь привести первую строку к виду второй с конца. В рекурсивной части функции первая строка символизирует вставку символа в первую строку из второй, ваторая строка символизирует удаление символа из первой, а последняя замену символа первой на вторую, при этом если символы уже равны, то замену проводить не нужно и операция стоит 0.

\subsection{Расстояние Левенштейна с кешированием}
Рекурсивная форма может оказаться малоэффективной при больших значениях N и M, так как в ней могут постоянно пересчитываться значения $D(i, j)$ для одних и тех же аргументов. Для устранения потенциального недостатка можно использовать итерационный алгоритм, который сохраняет промежуточные значения в виде матрицы размерности $(N+1)x(M+1)$.

Значения в ячейке [i, j] матрицы содержат значение $D(i, j)$, то есть расстояние левенштейна между подстроками $S_1[1...i]$ и $S_2[1...j]$.
При этом нулевая строка и нулевой столбец матрицы заполняются слева-направо и сверху-вниз от 0 до соотвествующей размерности,  так как приведение пустой строки к любой строке длины, скажем, i требует i операций вставки.

Далее алгоритм построчно заполняет матрицу по формуле (\ref{eq:D}), при этом вместо просчитывания предыдущих значений D используются значения из матрицы в соотвествующих ячейках.

Данный алгоритм может работать эффективнее по времени, однако за счёт хранения дополнительной матрицы может тратить больше памяти. Можно оптимизировать это, если на каждой итерации хранить только текущую строку матрицы и предыдущую, так как остальные не нужны для просчёта.

\section{Расстояние Дамерау-Левенштейна}

\textbf{Расстояние Дамерау-Левенштайна} - метрика, расстояние между двумя последовательностями символов, определяемая минимальным количеством операций вставки, замены, удаления и транспозиции символов, необходимых для преобразования одной последовательности в другую. Под транспозицией понимается перестановка двух соседних символов. От расстояния Левенштейна отличается тем, что вводится операция транспозиции, которая может существенно уменшить значение расстояния.

\subsection{Рекурсивная формула расстояния Дамерау-Левенштейна}

Введём также как и для расстояния Левенштейна функцию $D(i, j)$, при этом с учётом транспозиции формула будет выглядеть так:

\begin{equation}
	\label{eq:Dd}
	D(i, j) = 
	\begin{cases}
		0 &\text{i = 0 и j = 0}\\
		i &\text{i > 0 и j = 0}\\
		j &\text{i = 0 и j > 0}\\
		min \begin{cases}
			D(i, j - 1) + 1 &\text{Вставка}\\
			D(i - 1, j) + 1 &\text{Удаление}\\
			D(i - 1, j - 1) + f_e(S_1[i], S_2[j]) &\text{Замена}\\
			D(i - 2, j - 2) + 1 &\text{Транспозиция}
		\end{cases} & \begin{aligned}
		&i > 1 \ и\  j > 1 \ и\ \\ 
		&S_1[i] == S_2[j - 1] \ и \\
		&S_1[i - 1] == S_2[j]\\
		\end{aligned} \\
		min \begin{cases}
			D(i, j - 1) + 1 &\text{Вставка}\\
			D(i - 1, j) + 1 &\text{Удаление}\\
			D(i - 1, j - 1) + f_e(S_1[i], S_2[j]) &\text{Замена}\\
		\end{cases} &\text{Иначе}\\
	\end{cases}
\end{equation}

Фактическое отличие от алгоритма расстояния Левенштейна в том, что при j > 1 и i > 1, а также при условии что соседние символы в строках равны "накрест", может быть проведена операция транспозиции, которая может сократить расстояние.

\subsection{Расстояние Дамерау-Левенштейна с кешированием}
Также как и в случае расстояния Левенштейна, рекурсивный алгоритм может быть малоэффективным, поэтому имеет место быть итерационный алгоритм с кешированием, используя матрицу, либо текущую и последние две строки матрицы. Алгоритм аналогичен алгоритму расстояния, за исключением того, что при равенстве накрест двух соседних символов строк и значению i>1 и j>1 при расчёте очередной ячейки нужно проверить ещё и вариант с транспозицией.


\section*{Вывод}
В результате аналитического раздела были рассмотрены расстояния Левенштейна и Дамерау-Левенштейна, а также их рекурсивные формулы, итерационные алгоритмы с кешированием матрицами и отдельными строками матриц.

\clearpage
