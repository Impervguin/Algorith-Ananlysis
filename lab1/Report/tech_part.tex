\chapter{Технологическая часть}
\section{Средства разработки}

В качестве языка программирования был выбран python3 \cite{python3}, так как в его стандартной библиотеке присутсвуют функции замера процессорного времени, которые требуются в условиях, а также данный язык обладающий множеством инструментов для визуализации и работы с данными и таблицами.

Для основного файла был выбран инструмент jupyter notebook \cite{python3-jupyter}, так как он позволяет организовать код в удобные блоки,  а также выводить данные и графики прямо в нём, что позволяет легко продемонстрировать все замеры.

Для построения графиков использовалась библиотека plotly  \cite{python3-plotly}.

Для замера времени использовалась функция process\_time\_ns из стандартного модуля time \cite{python3-time}.

Для замеров памяти использовалась функция get\_traced\_memory стандартного модуля tracemalloc \cite{tracemalloc}, которая получает текущую и пиковую выделенную память относительно начала замера.

\section{Реализация алгоритмов}

В листингах \ref{recursive-levenshtein}–\ref{cache-damerau-levenshtein} приведены реализации разработанных в конструкторской части алгоритмов(рисунки \ref{fig:RL}–\ref{fig:RDLC2}).

\begin{lstlisting}[label=recursive-levenshtein,caption={Рекурсивный алгоритм нахождения расстояния Левенштейна}]
def RecursiveLevenshtein(s1, s2):
	if len(s1) == 0:
		return len(s2)
	if len(s2) == 0:
		return len(s1)
	if s1[0] == s2[0]:
		return RecursiveLevenshtein(s1[1:], s2[1:])
		return 1 + min(
	RecursiveLevenshtein(s1[1:], s2),
	RecursiveLevenshtein(s1[1:], s2[1:]),
	RecursiveLevenshtein(s1, s2[1:])
	)
\end{lstlisting}

\begin{lstlisting}[label=cache-levenshtein,caption={Итерационный алгоритм нахождения расстояния Левенштейна с кешэм}]
	def CacheLevenshtein(s1, s2):
		cacheRows = len(s1) + 1
		cacheCols = len(s2) + 1
		cache = [[0] * cacheCols for _ in range(cacheRows)]
		for i in range(1, cacheCols):
			cache[0][i] = i
		for i in range(1, cacheRows):
			cache[i][0] = i
		
		for i in range(1, cacheRows):
			for j in range(1, cacheCols):
				cache[i][j] = min(cache[i - 1][j] + 1, cache[i - 1][j - 1]  + (0 if s1[i - 1] == s2[j - 1] else 1), cache[i][j - 1] + 1)
		
		return cache[cacheRows - 1][cacheCols - 1]
	)
\end{lstlisting}

\begin{lstlisting}[label=cache-damerau-levenshtein,caption={Итерационный алгоритм нахождения расстояния Дамерау-Левенштейна с кешэм}]
	def CacheDamerauLevenshtein(s1, s2):
		cacheRows = len(s1) + 1
		cacheCols = len(s2) + 1
		cache = [[0] * cacheCols for _ in range(cacheRows)]
		for i in range(1, cacheCols):
			cache[0][i] = i
		for i in range(1, cacheRows):
			cache[i][0] = i
		for i in range(1, cacheRows):
			for j in range(1, cacheCols):
				if i >= 2 and j >= 2 and s1[i - 1] == s2[j - 2] and s1[i - 2] == s2[j - 1]:
					cache[i][j] = min(
						cache[i - 1][j] + 1,
						cache[i - 1][j - 1] + (0 if s1[i - 1] == s2[j - 1] else 1),
						cache[i - 2][j - 2] + 1,
						cache[i][j - 1] + 1,
						)
				else:
					cache[i][j] = min(
						cache[i - 1][j] + 1,
						cache[i - 1][j - 1] + (0 if s1[i - 1] == s2[j - 1] else 1),
						cache[i][j - 1] + 1,
						)
		
		return cache[cacheRows - 1][cacheCols - 1]
\end{lstlisting}

\section{Функциональные тесты}

В табличке \ref{tbl:func_tests} приведены тесты для алгоритмов поиска расстояние Левенштейна и Дамерау–Левенштейна.

\begin{table}[ht]
	\small
	\begin{center}
		\begin{threeparttable}
			\caption{Функциональные тесты}
			\label{tbl:func_tests}
			\begin{tabular}{|c|c|c|c|c|c|c|}
				\hline
				\multicolumn{2}{|c|}{\bfseries Входные данные}
				& \multicolumn{5}{c|}{\bfseries Расстояния и алгоритм} \\ 
				\hline
				&
				&
				\multicolumn{3}{c|}{Левенштейн} & \multicolumn{2}{c|}{Дамерау–Левенштейн} \\ \cline{3-7}
				Строка 1 & Строка 2 & Рекурсивный & С кешем &  Ожидаемое & С кешем & Ожидаемое \\
				\hline
				дмитрий & андрей & 5 & 5 & 5 & 5 & 5 \\
				\hline
				река & мука & 2 & 2 & 2 & 2 & 2 \\
				\hline
				1234 & 2134 & 3 & 3 & 3 & 2 & 2 \\
				\hline
			\end{tabular}	
		\end{threeparttable}
	\end{center}
\end{table}

Все тесты пройдены успешно

\section*{Вывод}

В ходе технологической части работы были разработаны алгоритмы поиска расстояния Левенштейна и Дамерау–Левенштейна, а также проведено их тестирование.

\clearpage
