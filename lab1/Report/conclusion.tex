\ssr{ЗАКЛЮЧЕНИЕ}

В результате исследования было выявлено, что сложность алгоритма поиска расстояния Левенштейна быстро растёт при увеличении длины строк. При этом итерационная реализация алгоритм с кешированием на несколько порядков быстрее рекурсивного варианта, за счёт того, что не пересчитывает заново значения для подстрок, как в рекурсивной реализации. С точки зрения ёмкостных характеристик, то при маленьких значениях длины строк($\leq 8$) выигрывает рекурсия, однако при длинах ($\ge 8$) итерационный вариант оказывается эффективнее.

Цель и все задачи лабораторной работы были выполнены.