\ssr{ЗАКЛЮЧЕНИЕ}

В ходе данной лабораторной работы были изучены метода динамического программирования на материале алгоритмов
Левенштейна и Дамерау-Левенштейна. 

Были выполнены следующие задачи:
\begin{enumerate}
	\item Теоритически были рассмотрены методы нахождения расстояния Левенштейна и Дамерау-Левенштейна;
	\item Были разработаны рекурсивный алгоритма расстояния Левенштейна, итерационные алгоритмы расстояний Левенштейна и Дамерау-Левенштейна с использованием методов динамического программирования, в частности с кешированием; 
	\item Реализованы указанные алгоритмы: два алгоритма с кешированием и один алгоритм в рекурсивной версии; 
	\item Было проведено исследование временных и емкостных характеристик реализованных алгоритмов на строка варьируемой длины.
\end{enumerate}

Цель и все задачи лабораторной работы были выполнены.
