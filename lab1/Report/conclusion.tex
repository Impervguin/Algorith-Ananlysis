\ssr{ЗАКЛЮЧЕНИЕ}

Цель - изучить методы динамического программирование на примере алгоритмов расстояния Левенштейна и Дамерау-Левенштейна - была выполнена.
При этом в ходе работы были выполнены следующие задачи:

\begin{itemize}
	\item Рассмотрены алгоритмы Левенштейна и Дамерау-Левенштейна нахождения расстояния между строками;
	\item Разработаны рекурсивный алгоритм расстояния Левенштейна и итерационные алгоритмы расстояний Левенштейна и Дамерау-Левенштейна с применением методов динамического программирования; 
	\item Реализованы разработанные алгоритмы на языке python3; 
	\item Проведено исследования различий в ёмкостной и временной характеристиках между рекурсивной и итерационной реализаций алгоритмов расстояния Левенштейна.
\end{itemize}

~\\

В результате исследования было выявлено, что сложность алгоритма поиска расстояния Левенштейна быстро растёт при увеличении длины строк. При этом итерационная реализация алгоритм с кешированием на несколько порядков быстрее рекурсивного варианта, за счёт того, что не пересчитывает заново значения для подстрок, как в рекурсивной реализации. С точки зрения ёмкостных характеристик, то при маленьких значениях длины строк($\leq 8$) выигрывает рекурсия, однако при длинах ($\ge 8$) итерационный вариант оказывается эффективнее.

