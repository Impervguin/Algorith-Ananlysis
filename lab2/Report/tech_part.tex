\chapter{Технологическая часть}
\section{Средства разработки}

В качестве языка программирования был выбран python3 \cite{python3}, так как в его стандартной библиотеке присутствуют функции замера процессорного времени, которые требуются в условиях, а также данный язык обладает множеством инструментов для работы с данными.
В частности была взята его реализация micropython \cite{micropython}, которая разработана для работы с микроконтроллерами, на которых планировалось проводить замеры.

Для файла с графиком был выбран инструмент jupyter notebook \cite{python3-jupyter}, так как он позволяет организовать код в удобные блоки,  а также выводить данные и графики прямо в нём, что позволяет легко продемонстрировать все замеры.

Для построения графиков использовалась библиотека plotly \cite{python3-plotly}.

Для замера времени использовалась функция ticks\_ms() из стандартного модуля utime \cite{python3-utime} для micropython.

\section{Реализация алгоритмов}

В листингах \ref{stdpy}–\ref{vinpy} приведены реализации разработанных в конструкторской части алгоритмов(рисунки \ref{standard}–\ref{ovinpy}).

\begin{lstlisting}[label=stdpy,caption={Стандартный алгоритм умножения матриц}]
	def SimpleMatrixMultiply(m1: list[list[int]], m2: list[list[int]]):
		if len(m1[0]) != len(m2):
			raise ValueError("Matrices cannot be multiplied")
		
		result = [[0] * len(m2[0]) for _ in range(len(m1))]
		
		for i in range(len(m1)):
			for j in range(len(m2[0])):
				for k in range(len(m2)):
					result[i][j] += m1[i][k] * m2[k][j]
		
		return result
\end{lstlisting}

\begin{lstlisting}[label=vinpy,caption={Алгоритм умножения матриц Винограда}]
	def VinogradMatrixMultiply(m1: list[list[int]], m2: list[list[int]]):
		if (len(m1) == 0 or len(m2) == 0):
			raise ValueError("Empty matrix")
		
		if len(m1[0]) != len(m2):
			raise ValueError("Matrices cannot be multiplied")
		
		M = len(m1)
		N = len(m1[0]) # == len(m2)
		Q = len(m2[0])
		result = [[0] * Q for _ in range(M)]
		
		mulH = [0] * (M)
		for i in range(M):
			for j in range(N // 2):
				mulH[i] = mulH[i] + m1[i][2*j] * m1[i][2*j + 1]
		
		mulV = [0] * (Q)
		for i in range(Q):
			for j in range(N // 2):
				mulV[i] = mulV[i] + m2[2*j][i] * m2[2*j + 1][i]
		
		for i in range(M):
			for j in range(Q):
				result[i][j] = -mulH[i] -mulV[j]
				for k in range(N // 2):
					result[i][j] = result[i][j] + (m1[i][2*k]+m2[2*k + 1][j]) * (m1[i][2*k + 1]+m2[2*k][j])
		
		if (N % 2 != 0):
			for i in range(M):
				for j in range(Q):
					result[i][j] = result[i][j] + m1[i][-1] * m2[-1][j]
		
		return result
\end{lstlisting}

\begin{lstlisting}[label=ovinpy,caption={Оптимизированный алгоритм умножения матриц Винограда}]
	def OptimizedVinogradMatrixMultiply(m1: list[list[int]], m2: list[list[int]]):
		if (len(m1) == 0 or len(m2) == 0):
			raise ValueError("Empty matrix")
		
		if len(m1[0]) != len(m2):
			raise ValueError("Matrices cannot be multiplied")
		
		M = len(m1)
		N = len(m1[0]) # == len(m2)
		Q = len(m2[0])
		result = [[0] * Q for _ in range(M)]
		
		mulH = [0] * (M)
		for i in range(M):
			mulH[i] = m1[i][0] * m1[i][1]
			for j in range(2, N - 1, 2):
				mulH[i] += m1[i][j] * m1[i][j + 1]
		
		mulV = [0] * (Q)
		for i in range(Q):
			mulV[i] = m2[0][i] * m2[1][i]
			for j in range(2, N - 1, 2):
				mulV[i] += m2[j][i] * m2[j + 1][i]
		
		for i in range(M):
			for j in range(Q):
				result[i][j] = -mulH[i] -mulV[j] + (m1[i][0]+m2[1][j]) * (m1[i][1]+m2[0][j])
				for k in range(2, N - 1, 2):
					result[i][j] += (m1[i][k]+m2[k + 1][j]) * (m1[i][k + 1]+m2[k][j])
		
		if (N % 2 != 0):
			for i in range(M):
				for j in range(Q):
					result[i][j]  += m1[i][-1] * m2[-1][j]
		
		return result
\end{lstlisting}


\section*{Вывод}

В ходе технологической части работы были реализованы алгоритмы умножения матриц разработанные в конструкторской части.

\clearpage
