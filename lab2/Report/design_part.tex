\chapter{Конструкторская часть}

\section{Требования к программному обеспечению}

К разрабатываемой программе предъявлен ряд требований.

\textbf{Входные данные:} Две матрицы, подходящие для умножения размеров

\textbf{Выходные данные:} Матрица, являющаяся их произведением

\section{Разработка алгоритмов}
Пусть в качестве входных данных алгоритмам подаются  матрицы A и B размерами M на N и N на Q элементов соответственно. На листинге \ref{standard} показан псевдокод стандартного алгоритма умножения матриц, на листинге \ref{vinograd} – алгоритм винограда.

\begin{lstlisting}[label=standard, caption={Псевдокод стандартного алгоритма умножения}]
for (i=0; i < M; i++) {
	for (j=0; j < Q; j++) {
		C[i][j] = 0                  
		for (k=0; k < N; k++) {
			C[i][j] = C[i][j] +      
			A[i][k]*B[k][j];         
		}
	}
}

\end{lstlisting}

\begin{lstlisting}[label=vinograd, caption={Псевдокод алгоритма Винограда}]
	// I. Заполнение массива mulH
	for (i=0; i < M; i++) {
		for (k = 0; k < N/2; k++) {
			mulH[i] = mulH[i] +         
			A[i][2*k] * A[i][2*k + 1]   
		}
	}
	
	// II. Заполнение mulV
	for (i=0; i < Q; i++) {
		for (k = 0; k < N/2; k++) {
			mulV[i] = mulV[i] +         
			B[2*k][i] * B[2*k + 1][i]  
		}
	}
	
	
	// III Основная часть
	for (i=0; i < m; i++) {
		for (j=0; j < Q; j++) {
			C[i][j] = -Mulh[i] - MulV[j] 
			for (k=0; k < N/2; k++) {
				C[i][j] = C[i][j] +       
				(A[i][2*k]+B[2*k + 1][j]) *
				(A[i][2*k + 1]+B[2*k][j]);  
			}
		}
	}
	
	// IV Обработка на случай нечётной общей размерности матриц
	if (N % 2 != 0) {
		for (i=0; i < m; i++) {
			for (j=0; j < Q; j++) {
				C[i][j] = C[i][j] + A[i][N-1] * B[N-1][j]
			}
		}
	}
\end{lstlisting}

В виде, представленном на листинге \ref{vinograd}, используется много неэффективных решений, в частности возможны следующие оптимизации:
\begin{itemize}
	\item во вложенных циклах замена шага на 2, а условие цикла до N;
	\item замена присваивания со сложением в телах циклов на инкремент; 
	\item в основной части, в самом вложенном цикле вынесение первой итерацию во вне цикла.
\end{itemize}

На листинге \ref{vinograd-optimize} можно увидеть псевдокод алгоритма Винограда с описанными выше оптимизациями.

\begin{lstlisting}[label=vinograd-optimize, caption={Псевдокод алгоритма Винограда с оптимизациями}]
	// I. Заполнение массива mulH
	for (i = 0; i < M; i++) {
		for (k = 0; k < N; k += 2) {
			mulH[i] += A[i][k] * A[i][k + 1]   
		}
	}
	
	// II. Заполнение mulV
	for (i = 0; i < Q; i++) {
		for (k = 0; k < N; k += 2) {
			mulV[i] += B[k][i] * B[k + 1][i]  
		}
	}
	
	
	// III Основная часть
	for (i = 0; i < m; i++) {
		for (j = 0; j < Q j++) {
			C[i][j] = -Mulh[i] - MulV[j] + (A[i][0]+B[1][j]) * (A[i][1]+B[0][j])
			for (k = 2; k < N; k += 2) {
				C[i][j] += (A[i][k]+B[k + 1][j]) * (A[i][k + 1]+B[k][j]);  
			}
		}
	}
	
	// IV Обработка на случай нечётной общей размерности матриц
	if (N % 2 != 0) {
		for (i=0; i < m; i++) {
			for (j=0; j < Q j++) {
				C[i][j] += A[i][N-1] * B[N-1][j]
			}
		}
	}
\end{lstlisting}

\section{Модель вычислений}
Для оценки трудоёмкости разработанных алгоритмов введём модель вычислений.

\subsection{Трудоёмкость базовых операций}
Примем единичной трудоёмкость следующих операций: =, +, -, +=, -=, ==, !=, <, <=, >=, >, [], <<, >>.
Трудоёмкость 2 имеют следующие операции: * /, *=, /=, \%.


\subsection{Трудоёмкость цикла}

Пусть для цикла вида:
\begin{equation}
	\label{eq:cycle}
	\begin{matrix*}[l]
		\text{for (инициализация; сравнение; инкремент) \{} \\
		\text{\ \ \ \ тело цикла}  \\
		\text{\},}
	\end{matrix*}
\end{equation}
известны трудоёмкости блоков инициализации, сравнения, инкремента и тела и они соответственно равны $f_{init}, f_{comp}, f_{inc}, f_{body}$.
Тогда трудоёмкость цикла рассчитывается по формуле \ref{eq:model-cycle}:
\begin{equation}
	\label{eq:model-cycle}
	f_{cycle} = f_{init} + f_{comp} + M(f_{comp} + f_{inc} + f_{body}),
\end{equation}
где M - количество итераций выполненных циклом.

\subsection{Трудоёмкость условного оператора}

Пусть для условного оператора вида:
\begin{equation}
	\label{eq:if}
	\begin{matrix*}[l]
		\text{if (условие) \{} \\
		\text{\ \ \ \ тело1}  \\
		\text{\} else \{} \\
		\text{\ \ \ \ тело2} \\
		\text{\},}
	\end{matrix*}
\end{equation}

известны трудоёмкости блоков условия, тел 1 и 2 и они соответственно равны $f_{cond}, f_{1}, f_{2}$.
Тогда трудоёмкость условного оператора рассчитывается по формуле \ref{eq:model-if}:
\begin{equation}
	\label{eq:model-if}
	f_{if} = f_{cond} + \begin{cases}
		min(f1, f2), &\text{в лучшем случае}, \\
		max(f1, f2), &\text{в худшем случае}.
	\end{cases}
\end{equation}

\section{Расчёт трудоёмкости алгоритмов}

Трудоёмкости для разработанных алгоритмов будем рассматривать в лучшем и худшем случаях относительно их размеров  M, N и Q.

Для стандартного алгоритма худший случай совпадает с лучшим, так как в нём нет условных операторов. Его трудоёмкость считается следующим образом:
\begin{equation}
	\label{eq:f-std}
	\begin{matrix}
		f_{std} = 14MNQ + 7MQ + 4M + 2
	\end{matrix}
\end{equation}

Слагаемые приведены в формуле \ref{eq:f-std} в порядке следования в программе \ref{standard}.

Для алгоритма винограда худшим случаем будет, когда общая размерность матриц будет нечётной, так как тогда будет дополнительно выполняться тело 4-й части программы \ref{vinograd}. Расчёт трудоёмкости всех частей алгоритма винограда приведён на формулах \ref{eq:vin1}–\ref{eq:vin4}.

\begin{equation}
	\label{eq:vin1}
	\begin{matrix}
		f_{I} = \frac{19}{2}MN + 6M + 2 
	\end{matrix}
\end{equation}

2-я часть программы \ref{vinograd} от 1-ой с точки зрения трудоёмкости отличается только переменной внешнего цикла, поэтому:

\begin{equation}
	\label{eq:vin2}
	\begin{matrix}
		f_{II} = \frac{19}{2}QN + 6Q + 2 
	\end{matrix}
\end{equation}

\begin{equation}
	\label{eq:vin3}
	\begin{matrix}
		f_{III} = 16MNQ + 13MQ + 4M + 2
	\end{matrix}
\end{equation}

\begin{equation}
	\label{eq:vin4}
	\begin{matrix}
		f_{IV} = \begin{cases}
			3, &\text{если N – чётно}, \\
			16MQ + 4M + 5, &\text{если N – нечётно}.
		\end{cases}
	\end{matrix}
\end{equation}

Тогда общая трудоёмкость алгоритма Винограда будет иметь вид:

\begin{equation}
	\label{eq:vin}
	\begin{matrix}
		f_{V} = \begin{cases}
			16MNQ + 13MQ + \frac{19}{2}N(M+Q) + 10M + 6Q + 9, &\text{если N – чётно}, \\
			16MNQ + 29MQ + \frac{19}{2}N(M+Q) + 14M + 6Q + 11, &\text{если N – нечётно}.
		\end{cases}
	\end{matrix}
\end{equation}

Порядок трудоёмкости, то есть сложность алгоритма, определяется через самое быстрорастущее слагаемое, то есть и для стандартного и для алгоритма Винограда это MNQ, при это коэффициент при этом слагаемом в алгоритме винограда выше, а значит он менее эффективен.

Рассмотрим оптимизированный алгоритм по частям, аналогично обычному алгоритму Винограда. Трудоёмкости приведены на \ref{eq:ovin1}–\ref{eq:ovin4}

\begin{equation}
	\label{eq:ovin1}
	\begin{matrix}
		f^o_{I} = \frac{11}{2}MN + 4M + 2 
	\end{matrix}
\end{equation}

\begin{equation}
	\label{eq:ovin2}
	\begin{matrix}
		f^o_{II} = \frac{11}{2}QN + 4Q + 2 
	\end{matrix}
\end{equation}

\begin{equation}
	\label{eq:ovin3}
	\begin{matrix}
		f^o_{III} = \frac{19}{2}MNQ + 5MQ + 4M + 2
	\end{matrix}
\end{equation}

\begin{equation}
	\label{eq:ovin4}
	\begin{matrix}
		f^o_{IV} = \begin{cases}
			3, &\text{если N – чётно}, \\
			13MQ + 4M + 5, &\text{если N – нечётно}.
		\end{cases}
	\end{matrix}
\end{equation}

Итоговая трудоёмкость оптимизированного алгоритма:

\begin{equation}
	\label{eq:ovin}
	\begin{matrix}
		f^o_{V} = \begin{cases}
			\frac{19}{2}MNQ + 5MQ + \frac{11}{2}N(M+Q) + 8M + 4Q + 9, &\text{если N – чётно}, \\
			\frac{19}{2}MNQ + 18MQ + \frac{11}{2}N(M+Q) + 12M + 4Q + 11, &\text{если N – нечётно}.
		\end{cases}
	\end{matrix}
\end{equation}

Таким образом оптимизированный алгоритм винограда должен быть эффективнее стандартного

\section{Вывод}

В результате конструкторской части были определены требования к ПО, а также разработаны псевдокоды алгоритмов стандартного умножения, умножения Винограда, проведена оптимизация умножения Винограда и рассчитаны трудоёмкости алгоритмов.

\clearpage
