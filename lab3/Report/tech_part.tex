\chapter{Технологическая часть}
\section{Средства разработки}

В качестве языка программирования был выбран python3 \cite{python3}, так как данный язык обладает множеством инструментов для визуализации данных и работы с ними.

Для основного файла был выбран инструмент jupyter notebook \cite{python3-jupyter}, так как он позволяет организовать код в удобные блоки,  а также выводить данные и графики прямо в нём, что позволяет легко продемонстрировать все исследования.

Для построения графиков использовалась библиотека plotly \cite{python3-plotly}.

\section{Реализация алгоритмов}

В листинге \ref{linear-search} представлена реализация алгоритма линейного поиска, а в листинге \ref{binary-search} – реализация бинарного поиска. В каждый из этих алгоритмов был добавлен счётчик сравнений в теле цикла, при этом в линейном поиске это счётчик увеличивает на 1 за итерацию, а в бинарном поиске может увеличиться на 1 или 2, в зависимости от отношения текущего элемента с искомым.

\begin{lstlisting}[label=linear-search,caption={Алгоритм линейного поиска}]
	def SimpleSearch(array: list[int], element: int) -> tuple[int, int]:
		comparisonCount = 0
		for i in range(len(array)):
			comparisonCount += 1
			if array[i] == element:
				return i, comparisonCount
	return -1, comparisonCount
\end{lstlisting}

\begin{lstlisting}[label=binary-search,caption={Алгоритм бинарного поиска}]
	def BinarySearch(array: list[int], element: int) -> tuple[int, int]:
		left = 0
		right = len(array) - 1
		comparisonCount = 0
	
		while left <= right:
			mid = (left + right) // 2
	
			if array[mid] < element:
				comparisonCount += 1
				left = mid + 1
			elif array[mid] > element:
				right = mid - 1
				comparisonCount += 2
			else:
				comparisonCount += 2
				return mid, comparisonCount
	
	return -1, comparisonCount
\end{lstlisting}

\section{Функциональные тесты}

При тестирования было выделено три вида тестов:
\begin{enumerate}
	\item Массив пустой;
	\item Искомого элемента в массиве нет;
	\item Искомы элемент есть в массиве.
\end{enumerate}

Для тестирования пунктов 1 и 2 были написаны отдельные тесты, а для пункта 3 использовалась следующая система: создавался массив целых чисел размером 1000 элементов с элемента $1,\dots,1000$ в порядке возрастания. Затем создавалась копия этого массива и перемешивалась случайным образом. Затем каждый элемент из перемешанной копии искался в исходном массиве линейным поиском и бинарным поиском, при этом значения функций сравнивались с методом list.index \cite{python3-list} из стандартной библиотеки python3, который ищет индекс элемента в массиве.

Для перемешивания массива была использована функция shuffle из стандартного модуля random \cite{python3-shuffle}.

Код тестирования приведён на листинге \ref{testing}.

\begin{lstlisting}[label=testing,caption={Тестирование функций поиска}]
testSize = 1000

array = [i for i in range(1, testSize + 1)]

# Поиск в пустом массиве
if SimpleSearch([], 2)[0] != -1:
	print("Error on empty array with SimpleSearch!")
	exit(1)

if BinarySearch([], 2)[0] != -1:
	print("Error on empty array with BinarySearch!")
	exit(1)

print("Empty array test passed!")

# Поиск несуществующего элемента
if SimpleSearch(array, testSize + 1000)[0] != -1:
	print("Error on not existing element with SimpleSearch!")
	exit(1)

if BinarySearch([], testSize + 1000)[0] != -1:
	print("Error on not existing element with BinarySearch!")
	exit(1)

print("Not existing element test passed!")

# Поиск всех элементов массива
arrcp = array.copy()
shuffle(arrcp)
for el in arrcp:
	index, _ = SimpleSearch(array, el)
    if index != array.index(el):
		print(f"Error on element {el} with SimpleSearch!")
		break
	index, _ = BinarySearch(array, el)
	if index != array.index(el):
		print(f"Error on element {el} with BinarySearch!")
		break
else:
	print("All positive tests passed!")
\end{lstlisting}

Все тесты пройдены успешно.

\section*{Вывод}

В ходе технологической части работы были разработаны алгоритмы линейного поиска и бинарного поиска на языке python3, а также проведено их функциональное тестирование.

\clearpage
