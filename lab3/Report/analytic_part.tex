\chapter{Аналитическая часть}
\section{Поиск полным перебором}
\textbf{Линейный поиск (полным перебором)} – алгоритм поиска в массиве, при котором массив последовательно просматривает поэлементно. Поиск прекращается при одном из двух условий: Искомый элемент найден по i-му индексу или не найден вовсе.

Линейный поиск не накладывает никаких дополнительных ограничений на массив, а значит универсален, так как можно применить к любому массиву, но он имеет сложность O(N) \cite{virt}, что является не самым эффективным решением.

\section{Бинарный поиск}
\textbf{Бинарный поиск (или поиск делением пополам)} – более эффективный по временной характеристике алгоритм для поиска элемента в отсортированном массиве. Ключевой идеей алгоритма является делением массива пополам на каждой итерации. Для этого берётся середина текущего массива и сравнивается с искомым элементом. Если искомый больше, то он точно правее текущего, иначе левее. За счёт этого получается на каждой итерации откидывать половину оставшегося массива.

Такой алгоритм имеет сложность O(logN) \cite{virt}, что уже лучше, чем линейный поиск, но такой алгоритм возможно применять только к отсортированным массивам, поэтому он не универсальный.


\section*{Вывод}
В результате аналитического раздела были рассмотрены алгоритмы линейного поиска и бинарного поиска в массивах.

\clearpage
